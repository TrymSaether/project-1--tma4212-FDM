\begin{remark}{Boundary Conditions}{}
    The boundary condition $g(x, t)$ is often given as a Dirichlet or Neumann condition, which specifies the temperature or heat flux at the boundary.
\end{remark}

\begin{remark}{Diffusion Term}{}
    The diffusion term $\mu \frac{\partial^2 u}{\partial x^2}$ describes how heat spreads through the material.
\end{remark}

\begin{remark}{Reaction Term}{}
    The reaction term $f(u)$ accounts for any heat sources or sinks in the system.
\end{remark}



% Extra consitency proof


We denote the exact, and intermediate PDE solution by
\[
  u_m^n = u(x_m, t_n), \quad u_m^\star = u(x_m, t_n + \Phi k)
\]
Where \(\Phi \in [0, 1]\) is a parameter that interpolates between the time levels \(t_n\) and \(t_{n+1}\).

We want to find the local truncation error (LTE) of the Crank-Nicolson scheme, which is the error at a single grid point when approximating the PDE with the finite difference scheme.

\[
  \norm{\tau_m^n} = \norm{u(x_m, t_n) - U_m^n} = \mathcal{O}(h^p + k^q) \quad \text{as} \quad h, k \to 0
\]

Then we substitute these expansions into the finite difference schemes to analyze the local truncation error (LTE):

\begin{align*}
  \delta_x^2 u_m^n & = u_{m+1}^n - 2 u_m^n + u_{m-1}^n                                                                                                    \\
                   & = \left[ u_m^n + h \partial_x u_m^n + \frac{h^2}{2} \partial_x^2 u_m^n + \frac{h^3}{6} \partial_x^3 u_m^n + \mathcal{O}(h^4) \right] \\
                   & - 2 u_m^n                                                                                                                            \\
                   & + \left[ u_m^n - h \partial_x u_m^n + \frac{h^2}{2} \partial_x^2 u_m^n - \frac{h^3}{6} \partial_x^3 u_m^n + \mathcal{O}(h^4) \right] \\
                   & = h^2 \partial_x^2 u_m^n + \mathcal{O}(h^4)
\end{align*}

\begin{align*}
  \delta_x^2 u_m^\star & = u_{m+1}^\star - 2 u_m^\star + u_{m-1}^\star                                                                                                                                                                                                                                 \\
                       & = \left[ u_m^n + \left(\Phi k \partial_t u_m^n + h \partial_x u_m^n\right) + \frac{1}{2}\left(\Phi^2 k^2 \partial_t^2 u_m^n + h^2 \partial_x^2 u_m^n\right) + \frac{1}{6}\left(\Phi^3 k^3 \partial_t^3 u_m^n + h^3 \partial_x^3 u_m^n\right) + \mathcal{O}(h^4 + k^4) \right] \\
                       & - 2 \left[ u_m^n + \Phi k \partial_t u_m^n + \frac{(\Phi k)^2}{2} \partial_t^2 u_m^n + \frac{(\Phi k)^3}{6} \partial_t^3 u_m^n + \mathcal{O}(k^4) \right]                                                                                                                     \\
                       & + \left[ u_m^n + \left(\Phi k \partial_t u_m^n - h \partial_x u_m^n\right) + \frac{1}{2}\left(\Phi^2 k^2 \partial_t^2 u_m^n + h^2 \partial_x^2 u_m^n\right) + \frac{1}{6}\left(\Phi^3 k^3 \partial_t^3 u_m^n - h^3 \partial_x^3 u_m^n\right) + \mathcal{O}(h^4 + k^4) \right] \\
                       & = h^2 \partial_x^2 u_m^n + \mathcal{O}(h^4 + k^4)                                                                                                                                                                                                                             \\
\end{align*}

Then we substitute these expansions into the Crank-Nicolson scheme to analyze the local truncation error (LTE):
\begin{align*}
  U_m^\star & = U_m^n + \frac{r}{2} \left( \delta_x^2 U_m^\star + \delta_x^2 U_m^n \right) + k a U_m^n                                  \\
            & = u_m^n + \frac{r}{2} \left( h^2 \partial_x^2 u_m^n + h^2 \partial_x^2 u_m^n \right) + k a u_m^n + \mathcal{O}(h^4 + k^4) \\
            & = u_m^n + r h^2 \partial_x^2 u_m^n + k a u_m^n + \mathcal{O}(h^4 + k^4)                                                   \\
\end{align*}

By substituting the expansions into the Crank-Nicolson scheme, we can analyze the local truncation error (LTE) of the scheme.

\begin{align*}
  U_m^{n+1} & = U_m^\star + \tfrac{k}{2} \Bigl( a\,U_m^\star - a\,U_m^n \Bigr)                                                                            \\[1mm]
            & = \Bigl[ u_m^n + k\,\partial_t u_m^n + \tfrac{k^2}{2}\,\partial_t^2 u_m^n + \tfrac{k^3}{6}\,\partial_t^3 u_m^n + \mathcal{O}(k^4) \Bigr]    \\
            & \quad {}+ \tfrac{k\,a}{2}\Bigl\{ \Bigl[ u_m^n + r\,h^2\,\partial_{xx} u_m^n + k\,a\,u_m^n
  + \mathcal{O}(h^4+k^2) \Bigr] - u_m^n \Bigr\}                                                                                                           \\[1mm]
            & = u_m^n + k\,\partial_t u_m^n + \tfrac{k^2}{2}\,\partial_t^2 u_m^n
  + \tfrac{k^3}{6}\,\partial_t^3 u_m^n + \mathcal{O}(k^4) + \tfrac{k\,a}{2}\Bigl( r\,h^2\,\partial_{xx} u_m^n + k\,a\,u_m^n \Bigr) + \mathcal{O}(h^4+k^4) \\[1mm]
            & = u_m^n + k\,\partial_t u_m^n + \tfrac{k^2}{2}\,\partial_t^2 u_m^n
  + \tfrac{k^3}{6}\,\partial_t^3 u_m^n
  + \tfrac{a\,r\,k}{2}\,h^2\,\partial_{xx} u_m^n + \tfrac{a^2\,k^2}{2}\,u_m^n
  + \mathcal{O}(k^4+h^4)
\end{align*}

Using \(r=\frac{\mu k}{h^2}\) so that \(r\,h^2=\mu k\), we rewrite the above as
\begin{align*}
  U_m^{n+1} & = u_m^n + k\,\partial_t u_m^n + \tfrac{k^2}{2}\,\partial_t^2 u_m^n
  + \tfrac{k^3}{6}\,\partial_t^3 u_m^n
  + \tfrac{a\,\mu\,k^2}{2}\,\partial_{xx} u_m^n + \tfrac{a^2\,k^2}{2}\,u_m^n
  + \mathcal{O}(k^4+h^4).
\end{align*}

The left-hand side of the scheme is the exact solution at the next time level:
\begin{align*}
  u_m^{n+1}
   & = u_m^n + k\,\partial_t u_m^n + \tfrac{k^2}{2}\,\partial_t^2 u_m^n
  + \tfrac{k^3}{6}\,\partial_t^3 u_m^n + \mathcal{O}(k^4).
\end{align*}

The local truncation error (LTE) is the difference between the exact solution and the numerical approximation:

\begin{align*}
  U_m^{n+1} - u_m^{n+1} & =
  \left[ u_m^n + k\,\partial_t u_m^n + \tfrac{k^2}{2}\,\partial_t^2 u_m^n + \tfrac{k^3}{6}\,\partial_t^3 u_m^n + \mathcal{O}(k^4) \right]                                                            \\
                        & - \left[ u_m^n + k\,\mu\,\partial_{x}^2 u_m^n + k\,a\,u_m^n + \tfrac{a\,\mu\,k^2}{2}\,\partial_{x}^2 u_m^n + \tfrac{a^2\,k^2}{2}\,u_m^n \right] + \mathcal{O}(k^4+h^4)     \\
                        & = k^2 \left[\tfrac{1}{2}\partial_t^2 u_m^n - \tfrac{a \,\mu}{2}\partial_{x}^2 u_m^n - \tfrac{a^2}{2}u_m^n\right] + \tfrac{k^3}{6}\partial_t^3 u_m^n + \mathcal{O}(k^4+h^4) \\
                        & = \mathcal{O}(k^2 + h^4)                                                                                                                                                   \\[2mm]
  \norm{\tau_m^n}       & = \norm{\frac{U_m^{n+1} - u_m^{n+1}}{k}} = \mathcal{O}(k + \frac{h^4}{k})                                                                                                  \\
\end{align*}

If we choose time step \(k\) such that \(k \sim h^2\) is of similar order as the spatial step, then the LTE is a second-order error in both time and space:

\begin{equation}
  \norm{\tau_m^n} = \mathcal{O}(k + \frac{h^4}{k}) \approx \mathcal{O}(k^2 + h^2) \quad \text{where} \quad k \sim h^2
\end{equation}