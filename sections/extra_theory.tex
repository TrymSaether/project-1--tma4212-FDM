\begin{remark}{Boundary Conditions}{}
  The boundary condition $g(x, t)$ is often given as a Dirichlet or Neumann condition, which specifies the temperature or heat flux at the boundary.
\end{remark}

\begin{remark}{Diffusion Term}{}
  The diffusion term $\mu \frac{\partial^2 u}{\partial x^2}$ describes how heat spreads through the material.
\end{remark}

\begin{remark}{Reaction Term}{}
  The reaction term $f(u)$ accounts for any heat sources or sinks in the system.
\end{remark}



% Extra consitency proof


We denote the exact, and intermediate PDE solution by
\[
  u_m^n = u(x_m, t_n), \quad u_m^\star = u(x_m, t_n + \Phi k)
\]

Where \(\Phi \in [0, 1]\) is a parameter that interpolates between the time levels \(t_n\) and \(t_{n+1}\).

We want to find the local truncation error (LTE) of the Crank-Nicolson scheme, which is the error at a single grid point when approximating the PDE with the finite difference scheme.

\[
  \norm{\tau_m^n} = \norm{u(x_m, t_n) - U_m^n} = \mathcal{O}(h^p + k^q) \quad \text{as} \quad h, k \to 0
\]

Then we substitute these expansions into the finite difference schemes to analyze the local truncation error (LTE):

\begin{align*}
  \delta_x^2 u_m^n & = u_{m+1}^n - 2 u_m^n + u_{m-1}^n                                                                                                    \\
                   & = \left[ u_m^n + h \partial_x u_m^n + \frac{h^2}{2} \partial_x^2 u_m^n + \frac{h^3}{6} \partial_x^3 u_m^n + \mathcal{O}(h^4) \right] \\
                   & - 2 u_m^n                                                                                                                            \\
                   & + \left[ u_m^n - h \partial_x u_m^n + \frac{h^2}{2} \partial_x^2 u_m^n - \frac{h^3}{6} \partial_x^3 u_m^n + \mathcal{O}(h^4) \right] \\
                   & = h^2 \partial_x^2 u_m^n + \mathcal{O}(h^4)
\end{align*}

\begin{align*}
  \delta_x^2 u_m^\star & = u_{m+1}^\star - 2 u_m^\star + u_{m-1}^\star                                                                                                                                                                                                                                 \\
                       & = \left[ u_m^n + \left(\Phi k \partial_t u_m^n + h \partial_x u_m^n\right) + \frac{1}{2}\left(\Phi^2 k^2 \partial_t^2 u_m^n + h^2 \partial_x^2 u_m^n\right) + \frac{1}{6}\left(\Phi^3 k^3 \partial_t^3 u_m^n + h^3 \partial_x^3 u_m^n\right) + \mathcal{O}(h^4 + k^4) \right] \\
                       & - 2 \left[ u_m^n + \Phi k \partial_t u_m^n + \frac{(\Phi k)^2}{2} \partial_t^2 u_m^n + \frac{(\Phi k)^3}{6} \partial_t^3 u_m^n + \mathcal{O}(k^4) \right]                                                                                                                     \\
                       & + \left[ u_m^n + \left(\Phi k \partial_t u_m^n - h \partial_x u_m^n\right) + \frac{1}{2}\left(\Phi^2 k^2 \partial_t^2 u_m^n + h^2 \partial_x^2 u_m^n\right) + \frac{1}{6}\left(\Phi^3 k^3 \partial_t^3 u_m^n - h^3 \partial_x^3 u_m^n\right) + \mathcal{O}(h^4 + k^4) \right] \\
                       & = h^2 \partial_x^2 u_m^n + \mathcal{O}(h^4 + k^4)                                                                                                                                                                                                                             \\
\end{align*}

Then we substitute these expansions into the Crank-Nicolson scheme to analyze the local truncation error (LTE):
\begin{align*}
  U_m^\star & = U_m^n + \frac{r}{2} \left( \delta_x^2 U_m^\star + \delta_x^2 U_m^n \right) + k a U_m^n                                  \\
            & = u_m^n + \frac{r}{2} \left( h^2 \partial_x^2 u_m^n + h^2 \partial_x^2 u_m^n \right) + k a u_m^n + \mathcal{O}(h^4 + k^4) \\
            & = u_m^n + r h^2 \partial_x^2 u_m^n + k a u_m^n + \mathcal{O}(h^4 + k^4)                                                   \\
\end{align*}

By substituting the expansions into the Crank-Nicolson scheme, we can analyze the local truncation error (LTE) of the scheme.

\begin{align*}
  U_m^{n+1} & = U_m^\star + \tfrac{k}{2} \Bigl( aU_m^\star - aU_m^n \Bigr)                                                                                                                                               \\
            & = \left[ u_m^n + k\partial_t u_m^n + \tfrac{k^2}{2}\partial_t^2 u_m^n + \tfrac{k^3}{6}\partial_t^3 u_m^n + \mathcal{O}(k^4) \right]                                                                      \\
            & \quad {}+ \tfrac{ka}{2}\left[\left( u_m^n + rh^2\partial_{xx} u_m^n + kau_m^n + \mathcal{O}(h^4+k^2) \right) - u_m^n \right]                                                                               \\
            & = u_m^n + k\partial_t u_m^n + \tfrac{k^2}{2}\partial_t^2 u_m^n + \tfrac{k^3}{6}\partial_t^3 u_m^n + \mathcal{O}(k^4) + \tfrac{ka}{2}\Bigl( rh^2\partial_{xx} u_m^n + kau_m^n \Bigr) + \mathcal{O}(h^4+k^4) \\
            & = u_m^n + k\partial_t u_m^n + \tfrac{k^2}{2}\partial_t^2 u_m^n
  + \tfrac{k^3}{6}\partial_t^3 u_m^n
  + \tfrac{ark}{2}h^2\partial_{xx} u_m^n + \tfrac{a^2k^2}{2}u_m^n
  + \mathcal{O}(k^4+h^4)
\end{align*}

The left-hand side of the scheme is the exact solution at the next time level:
\begin{align*}
  u_m^{n+1} & = u_m^n + k\partial_t u_m^n + \tfrac{k^2}{2}\partial_t^2 u_m^n + \tfrac{k^3}{6}\partial_t^3 u_m^n + \mathcal{O}(k^4).
\end{align*}

The local truncation error (LTE) is the difference between the exact solution and the numerical approximation:

\begin{align*}
  \abs*{\tau_m^{n+1}} & = \abs*{\dfrac{u_m^{n+1} - u_m^n}{k} - \dfrac{U_m^{n+1} - U_m^n}{k}} \\
                      & = \left[ u_m^n + k\partial_t u_m^n + \tfrac{k^2}{2}\partial_t^2 u_m^n + \tfrac{k^3}{6}\partial_t^3 u_m^n + \mathcal{O}(k^4) \right]                                      \\
                      & - \left[ u_m^n + k\mu\partial_{x}^2 u_m^n + kau_m^n + \tfrac{a\mu k^2}{2}\partial_{x}^2 u_m^n + \tfrac{a^2k^2}{2}u_m^n \right] + \mathcal{O}(k^4+h^4)                    \\
                      & = k^2 \left[\tfrac{1}{2}\partial_t^2 u_m^n - \tfrac{a \mu}{2}\partial_{x}^2 u_m^n - \tfrac{a^2}{2}u_m^n\right] + \tfrac{k^3}{6}\partial_t^3 u_m^n + \mathcal{O}(k^4+h^4) \\
                      & = \mathcal{O}(k^2 + h^4)                                                                                                                                                 \\
  \norm{\tau_m^n}     & = \norm{\frac{U_m^{n+1} - u_m^{n+1}}{k}} = \mathcal{O}(k + \frac{h^4}{k})                                                                                                \\
\end{align*}

If we choose time step \(k\) such that \(k \sim h^2\) is of similar order as the spatial step, then the LTE is a second-order error in both time and space:

\[
  \norm{\tau_m^n} = \mathcal{O}(k + \frac{h^4}{k}) \approx \mathcal{O}(k^2 + h^2) \quad \text{where} \quad k \sim h^2
\]



\paragraph{System of equations}

For a fixed time step \(n\), we can write the system of equations as
\begin{align*}
  \begin{pmatrix}
    S_{1,1}^{n+1}     \\
    S_{1,2}^{n+1}     \\
    \vdots            \\
    S_{M-1,M-1}^{n+1} \\
    I_{1,1}^{n+1}     \\
    I_{1,2}^{n+1}     \\
    \vdots            \\
    I_{M-1,M-1}^{n+1}
  \end{pmatrix}
  =
  \begin{pmatrix}
    A_{11} & A_{12} \\
    A_{21} & A_{22}
  \end{pmatrix}
  \begin{pmatrix}
    S_{1,1}^n     \\
    S_{1,2}^n     \\
    \vdots        \\
    S_{M-1,M-1}^n \\
    I_{1,1}^n     \\
    I_{1,2}^n     \\
    \vdots        \\
    I_{M-1,M-1}^n
  \end{pmatrix}
\end{align*}


The first matrix \(A_{11}\) is:
\small
\begin{align*}
  A_{11} & = \scriptscriptstyle
  \begin{bmatrix}
    1 - \beta S_{1,1}^n - \tfrac{4\mu_S}{h^2} & \tfrac{\mu_S}{h^2}                        & 0                                         & \tfrac{\mu_S}{h^2}                        & 0                  & \cdots                                        & 0      \\
    \tfrac{\mu_S}{h^2}                        & 1 - \beta S_{1,2}^n - \tfrac{4\mu_S}{h^2} & \tfrac{\mu_S}{h^2}                        & 0                                         & \tfrac{\mu_S}{h^2} & \cdots                                        & 0      \\
    0                                         & \frac{\mu_S}{h^2}                         & 1 - \beta S_{1,3}^n - \tfrac{4\mu_S}{h^2} & 0                                         & 0                  & \cdots                                        & 0      \\
    \tfrac{\mu_S}{h^2}                        & 0                                         & 0                                         & 1 - \beta S_{2,1}^n - \tfrac{4\mu_S}{h^2} & \tfrac{\mu_S}{h^2} & \cdots                                        & 0      \\
    \vdots                                    & \vdots                                    & \vdots                                    & \vdots                                    & \ddots             & \ddots                                        & \vdots \\
    0                                         & 0                                         & 0                                         & 0                                         & \cdots             & 1 - \beta S_{M-1,M-1}^n - \tfrac{4\mu_S}{h^2} & 0
  \end{bmatrix}
\end{align*}

The second matrix \(A_{22}\) is:
\begin{align*}
  A_{22} & = \scriptscriptstyle
  \begin{bmatrix}
    1 + \beta S_{1,1}^n - \gamma - \tfrac{4\mu_I}{h^2} & \tfrac{\mu_I}{h^2}                                 & 0                                                  & \tfrac{\mu_I}{h^2}                                 & 0                  & \cdots                                             & 0      \\
    \tfrac{\mu_I}{h^2}                                 & 1 + \beta S_{1,2}^n - \gamma - \tfrac{4\mu_I}{h^2} & \tfrac{\mu_I}{h^2}                                 & 0                                                  & \tfrac{\mu_I}{h^2} & \cdots                                             & 0      \\
    0                                                  & \tfrac{\mu_I}{h^2}                                 & 1 + \beta S_{1,3}^n - \gamma - \tfrac{4\mu_I}{h^2} & 0                                                  & 0                  & \cdots                                             & 0      \\
    \tfrac{\mu_I}{h^2}                                 & 0                                                  & 0                                                  & 1 + \beta S_{2,1}^n - \gamma - \tfrac{4\mu_I}{h^2} & \tfrac{\mu_I}{h^2} & \cdots                                             & 0      \\
    \vdots                                             & \vdots                                             & \vdots                                             & \vdots                                             & \ddots             & \ddots                                             & \vdots \\
    0                                                  & 0                                                  & 0                                                  & 0                                                  & \cdots             & 1 + \beta S_{M-1}^n - \gamma - \tfrac{4\mu_I}{h^2}
  \end{bmatrix}
\end{align*}

The matrices \(A_{12}\) and \(A_{21}\) are both zero matrices:
\begin{align*}
  A_{12} = A_{21} & =
  \begin{bmatrix}
    0      & 0      & 0      & \cdots & 0      \\
    0      & 0      & 0      & \cdots & 0      \\
    \vdots & \vdots & \vdots & \ddots & \vdots \\
    0      & 0      & 0      & \cdots & 0
  \end{bmatrix}
\end{align*}
\begin{tikzpicture}[
  auto matrix/.style={
  matrix of nodes,
  draw, thick,
  inner sep=0pt,
  nodes in empty cells,
  column sep=-0.2pt,
  row sep=-0.2pt,
  cells={
  nodes={
  minimum width=2em,
  minimum height=2em,
  draw,
  very thin,
  anchor=center,
  fill=white,
  execute at begin node={%
  $\vphantom{x_1^1}
    \pgfmathtruncatemacro{\itest}{sign(4-\the\pgfmatrixcurrentcolumn)*sign(4-\the\pgfmatrixcurrentrow)}
    \unless\ifnum\itest=0
    {#1}^{\myrowindex{\the\pgfmatrixcurrentrow}}_{\mycolindex{\the\pgfmatrixcurrentcolumn}}
    \fi
    \ifnum\the\pgfmatrixcurrentrow\the\pgfmatrixcurrentcolumn=14 \cdots\fi
    \ifnum\the\pgfmatrixcurrentrow\the\pgfmatrixcurrentcolumn=41 \vdots\fi
    \ifnum\the\pgfmatrixcurrentrow\the\pgfmatrixcurrentcolumn=44 \ddots\fi
  $}}}}]
  % Helper commands
  \newcommand{\mycolindex}[1]{\ifnum#1=5 M\else #1\fi}
  \newcommand{\myrowindex}[1]{\ifnum#1=5 N\else #1\fi}

  % Matrices for Susceptible population
  \matrix[auto matrix=S,xshift=5em,yshift=5em](matS){
     &  &  &  & \\
     &  &  &  & \\
     &  &  &  & \\
     &  &  &  & \\
     &  &  &  & \\
  };

  % Matrices for Infected population
  \matrix[auto matrix=I,xshift=-10em,yshift=0em](matI){
     &  &  &  & \\
     &  &  &  & \\
     &  &  &  & \\
     &  &  &  & \\
     &  &  &  & \\
  };

  % Arrows and labels
  \draw[thick,-stealth] ([xshift=0, yshift=-2ex]matS.south east) -- ([xshift=-28ex,yshift=-5em]matS.south east)
  node[midway,right] {$y$};
  \draw[thick,-stealth] ([yshift=-1ex]matS.south west) -- ([yshift=-1ex]matS.south east)
  node[midway,below] {$x$};
  \draw[thick,-stealth] ([xshift=-1ex]matS.north west) -- ([xshift=-1ex]matS.south west)
  node[midway,left] {$t$};

  % Labels for matrices
  \node[above] at (matS.north) {Susceptible};
  \node[above] at (matI.north) {Infected};
\end{tikzpicture}



\paragraph{Condition 2}
\begin{align*}
  \norm*{\xi}                & = \frac{(ka)^2 + (ka) + 1}{2} \leq 1 + \mu k                                                                       \\
  a^2 k^2 + (a - 2\mu) k - 1 & \leq 0 \quad \implies \quad k_{1,2} = \frac{-(a - 2\mu) \pm \overbrace{\sqrt{(a - 2\mu)^2 + 4a^2}}^{\geq 0}}{2a^2} \\
\end{align*}