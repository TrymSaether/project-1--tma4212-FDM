\chapter{Theoretical Background}

\begin{definition}{Local Truncation Error}{lte}
  The local truncation error (LTE) is the error made at a single grid point when approximating a differential equation with a finite difference scheme.
  \[
    \norm{\tau_m^n} = \norm{u(x_m, t_n) - U_m^n} = \mathcal{O}(h^p + k^q) \quad \text{as} \quad h, k \to 0 
  \]
  where \(p\) and \(q\) are the order of the spatial and temporal discretizations, respectively.
\end{definition}

\begin{lemma}{Consistency}{consistency}
  Suppose the solution \(u\) of \(u_t = \mu u_{xx} + a u\) is sufficiently smooth on the given space-time domain.
  Then the local truncation error of the above scheme is \(\mathcal{O}(k^2 + h^2)\), and thus the method is second-order accurate in both time and space.
  
  \[
    \norm{\tau_m^n} = \norm{u(x_m, t_n) - U_m^n} = \mathcal{O}(h^2 + k^2) \quad \text{as} \quad h, k \to 0
  \]

\end{lemma}




\paragraph{Notation}
\begin{tabular}{ll}
  \textbf{Symbol}               & \textbf{Meaning}                       \\
  \hline                                                                 \\[-1em]
  \(u = u(x, t)\)               & Exact temperature distribution         \\
  \(u_m^n = u(x_m, t_n)\)       & Exact temperature at grid point        \\
  \(\mu\)                       & Diffusion coefficient                  \\
  \(f(u)\)                      & Reaction term                          \\
  \(u_0(x)\)                    & Initial temperature distribution       \\
  \(g(x, t)\)                   & Boundary condition                     \\
  \(\Omega\)                    & Spatial domain                         \\
  \(\partial \Omega\)           & Boundary of spatial domain             \\
  \hline                                                                 \\[-1em]
  \(U_m^n \approx u(x_m, t_n)\) & Numerical approximation of temperature \\
  \(x_m = m h\)                 & Spatial grid points                    \\
  \(t_n = n k\)                 & Temporal grid points                   \\
  \(h = L/M\)                   & Spatial step size                      \\
  \(k =T/N\)                    & Temporal step size                     \\
  \(L\)                         & Spatial domain length                  \\
  \(T\)                         & Final time                             \\
  \(M\)                         & Number of spatial grid points          \\
  \(N\)                         & Number of temporal grid points         \\
  \(\mathbb{G}\)                & Set of all grid points                 \\
  \hline
\end{tabular}
