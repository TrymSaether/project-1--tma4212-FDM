\section{Application: Spatial-Temporal SIR Model}

We study the spatial-temporal spread of a disease by extending the classic SIR model to two spatial dimensions. 
This results in a reaction-diffusion system for \((S,I)\) on a square domain \(\Omega = [0,L]\times [0,L]\). As
discussed in the theory section this is a good candidate for the modified Crank--Nicolson scheme.

\subsection{Model Formulation and Modifications}

A classical SIR system tracks susceptible \(S\), infected \(I\), and removed \(R\). 
At a single location, one has
\[
  \begin{cases}
    S'(t) \;=\; -\beta\,S(t)\,I(t),\\[4pt]
    I'(t) \;=\; \beta\,S(t)\,I(t)\;-\;\gamma\,I(t),\\[4pt]
    R'(t) \;=\;\gamma\,I(t).
  \end{cases}
\]
Here \(\beta\) is the infection rate and \(\gamma\) the recovery rate. And \(S+I+R=1\) when normalized to a 
unit population.

To allow disease to spread across a region, we introduce diffusion of \(S\) and \(I\).
\[
  \begin{cases}
    S_t \;=\; -\beta\,S\,I \;+\;\mu_S\,\Delta S,\\[6pt]
    I_t \;=\; \beta\,S\,I \;-\;\gamma\,I \;+\;\mu_I\,\Delta I,
  \end{cases}
  \quad (x,y)\in\Omega,\;t>0.
\]
We set \(R=1-S-I\) and update it via \(R_t = \gamma\,I\). Here \(\Delta = \partial_x^2 + \partial_y^2\) is the 2D 
Laplacian, and \(\mu_S,\mu_I\) are diffusion coefficients.


Additionally to let this modell be better applicable to realistic scenarios we modify \(\beta\) to be dynamic in 
space and time, i.e.\ \(\beta = \beta(x,y,t)\). Physically, it might be higher in densely populated areas or 
vary over time for example periodic public gatherings.

\subsection{Numerical Implementation}

\subsubsection{Discretization of the Domain}

We subdivide \(\Omega=[0,L]\times [0,L]\) into an \(\!M\times M\) grid. Let \(h=L/M\) so that the grid points in
each direction are \(x_i = i\,h\), for \(i=0,\dots,M\). Combining,
\(\bigl\{(x_i, y_j)\bigr\}\) yields \(M^2\) internal variables for each unknown (\(S\) and \(I\)).

We approximate \(\Delta u\approx \partial_x^2u+\partial_y^2u\) by standard central differences. In one dimension 
(size \(M\)), the matrix for second differences is tridiagonal:
\[
  L_{\text{1D}} \;=\; \frac{1}{h^2}\,\begin{bmatrix}
    -2 & 1 &         &   &     \\
     1 & -2 & 1      &   &     \\
       & 1  & \ddots & \ddots &\\
       &    & \ddots & -2 & 1   \\
       &    &        & 1 & -2
  \end{bmatrix}.
\]
To handle a 2D Laplacian, we form \(L_{\text{2D}} = \text{kron}(I,L_{\text{1D}}) + \text{kron}(L_{\text{1D}},I)\), 
where \(\text{kron}\) is the Kronecker product and \(I\) is the \(M\times M\) identity. This yields an 
\((M^2)\times(M^2)\) sparse matrix. In the code, \(\mathrm{laplacian()}\) returns precisely this matrix.

\subsubsection{Chosen Boundary Conditions and Parameters}

We use a no-flux boundary condition, effectively, \(\mu \Delta u\) is discretized so that outward flux is zero 
at the domain edges. This is reflected by the modified diagonal entries for the topmost and bottommost rows 
in \(L_{\text{1D}}\). This is typical of a homogeneous Neumann implementation in central differences.


In the code, \(\beta=3\) and \(\gamma=1\) are chosen as baseline infection and recovery rates, matching simpler 
1D SIR examples.  We set \(\mu_S\) and \(\mu_I\) to small values so that susceptible and infected individuals do 
not diffuse too quickly.  The domain is \(L=1\), and we choose \(M=50\), 50 intervals in each dimension, and a 
small time-step \(\text{dt} = 0.001\).  The simulation runs up to \(T=10\) units of time.


For the initial conditions there are three modes of “infected” initialization:
\begin{itemize}
  \item \(\texttt{n=0}\): Dense local infection in corners, using a Gaussian peak near \(\tfrac{L}{4},\tfrac{L}{4}\) plus its mirror.
  \item \(\texttt{n=1}\): A linear front of infection occupying the leftmost 20\% of the domain.
  \item \(\texttt{n=2}\): Several random patches within the interior, mimicking local clusters.
\end{itemize}
We let \(S=1 - I\) initially (with \(R=0\)).  The choice of \( \approx 0.01\) sets how large the initial 
infection fraction is.

\subsubsection{Beta function}
As previously stated, the SIR model is modified when a dynamic beta, meaning that beta is a function dependent on 
both time and space. This is implemented via:
\[
  \beta(x,y,t) 
  \;=\; \beta_0 \times \bigl[1 + 0.5\,e^{-100\,((x-\tfrac{L}{2})^2 + (y-\tfrac{L}{2})^2)}\bigr]
  \times \bigl[1 + 0.1 \sin(t-2)\bigr]\quad (\text{for }2\le t \le 2+\pi),
\]
and equals the constant \(\beta_0\) otherwise.

\subsection{Implementation Details}

\paragraph{Sparse Matrix Storage.}
Since our 2D Laplacian is large \((M^2\times M^2)\) but banded, we store it in sparse format via 
\(\texttt{diags}\), \(\texttt{eye}\), and \(\texttt{kron}\). This greatly reduces memory usage and speeds up 
matrix operations.  Each time-step, we multiply this matrix by \(\mathrm{S}\) and \(\mathrm{I}\) to account for 
diffusion.

\paragraph{Time-Stepping.}
Given \(\Delta t=\text{dt}\),
\[
  S^{n+1} 
  \;=\;
  S^n + \Delta t\,\Bigl[\mu_S\,L_{\text{2D}}\,S^n - \beta(x,y,t_n)\,S^n\,I^n\Bigr],
  \quad
  I^{n+1}
  \;=\;
  I^n + \Delta t\Bigl[\mu_I\,L_{\text{2D}}\,I^n + \beta\,S^n\,I^n - \gamma\,I^n\Bigr].
\]
We also update 
\(
  R^{n+1} = R^n + \Delta t\,\bigl[\gamma\,I^n\bigr].
\)
If \(\beta\) is dynamic, we compute \(\beta(x,y,t)\) each step before forming the infection source term.

\subsection{Justification of Choices}

\begin{itemize}
  \item \textbf{Domain and Grid:} We use a unit square \((0,1)\times (0,1)\) with a \(50\times 50\) mesh for a compromise between resolution and runtime. A larger mesh would capture finer diffusion details but slow the simulation significantly.
  \item \textbf{Time Step:} We pick \(\Delta t = 0.001\) so that the PDE solution remains stable for the chosen parameters; in principle the method could allow larger \(\Delta t\), but high \(\beta\) and multiple infection clusters can demand refined time resolution for accurate capturing of fast infection waves.
  \item \textbf{Boundary Conditions:} A near--Neumann approach ensures people cannot exit or enter the domain, consistent with a self-contained population. 
  \item \textbf{Parameter Values:} Baseline \(\beta=3\) and \(\gamma=1\) produce a typical SIR behavior in well-known 1D test cases. Varying \(\mu_S\) and \(\mu_I\) influences how quickly each group moves about the domain.
  \item \textbf{Initial Infections:} We tested three distinct initial distributions (\texttt{n=0,1,2}) to see how infection geometry affects overall spread. The optional \(\texttt{dynamic\_beta}\) toggles a time- and space-dependent infection rate to model ephemeral gatherings or hotspots.
\end{itemize}

\subsection{Code Commentary}

\paragraph{Structure.}
The \(\texttt{SIRSimulation}\) class encapsulates domain setup, matrix assembly, initial conditions, and time-stepping (\(\texttt{step}\)). The 2D arrays \(S,I,R\) are flattened to vectors of length \(M^2\). 

\paragraph{Laplacian.}
The method \(\texttt{laplacian()}\) constructs the 1D second-difference operator as \(\texttt{L}\), then forms the 2D operator via Kronecker products. The diagonal near the boundary is adjusted so that the first and last rows have \(-1\) on the diagonal, consistent with no-flux or partial Neumann boundary approximations.

\paragraph{Integration Loop.}
At each step, we compute reaction terms \(\beta SI\) and \(\gamma I\), plus diffusion terms \(\mu_S L\cdot S\) and \(\mu_I L\cdot I\). We then do an explicit Euler update in time:
\[
  S \gets S + \mathrm{dt}\times(\text{diffusion} - \beta S I),\quad
  I \gets I + \mathrm{dt}\times(\text{diffusion} + \beta S I - \gamma I).
\]
An optional \(\beta\)-function can incorporate both a radial shape factor and a time-sinusoid, giving “hotspots” in space and fluctuating infection rates in time.

\paragraph{Animation.}
The \(\texttt{animate()}\) method updates in increments of 50 small steps (to keep the display from flickering). Matplotlib’s \(\texttt{FuncAnimation}\) drives the update loop, plotting the infected fraction in a 3D surface.

\subsection{Concluding Remarks on the SIR Simulation}

By combining a two-dimensional diffusion operator (for $S$ and $I$) with the usual SIR reaction terms, we capture how local infection spreads spatially. Our chosen domain and parameters illustrate various infection patterns:
\begin{itemize}
  \item \textbf{Small \(\mu_S,\mu_I\)}: keeps population mostly localized, leading to distinct infection clusters.
  \item \textbf{Modified \(\beta(x,y,t)\)}: represents spatiotemporal heterogeneity, e.g.\ higher infection rates in the central region or for certain times.
\end{itemize}
Though we used straightforward explicit Euler in time, the moderate diffusion and relatively small \(\Delta t\) suffice to maintain stability. More advanced time-stepping schemes (like Crank--Nicolson) could also be adapted for faster runs if necessary. Overall, our setup provides a flexible environment to explore how disease waves propagate and to visualize key features of spatio-temporal epidemics.
