\section{Discussion}

\subsection{Theory versus Numerical Results}
In principle, the linear stability analysis applies most directly if \(f(u)\) is linear. Our tests used a strongly 
nonlinear infection term \(\beta\,S\,I\), so the unconditional stability partially hinges on the moderate size of 
\(\beta\) and small time steps in practice. The local truncation argument still informs us that spatial errors 
scale like \(h^2\) and temporal errors like \(k^2\). Numerically, we see outcomes 
consistent with that prediction, but if one pushed the ratio \(k/h^2\) too high or introduced extremely large 
\(\beta\), the scheme could exhibit instability or large phase errors. Thus, while the theoretical 
\(\mathcal{O}(h^2)\) accuracy remains a useful guideline, the real PDE problem is more delicate than a purely 
linear analysis suggests.

\subsection{Limitations of Numerical Scheme}
Although our numerical scheme is supported by a consistency and stability analysis for linearized 
problems, it faces some caveats. First, the reaction-diffusion equation with a fully nonlinear 
reaction \(f(u)\) may require additional care, especially if the reaction term grows quickly or exhibits stiff 
behavior. In that case, even though the scheme remains conditionally consistent, larger time steps may trigger 
inaccuracies in the reaction component.

\subsection{Limitations of SIR Model}
On the modeling side, our SIR setup simplifies many real-world complexities. The population is mostly uniformly 
distributed except for the small adjustments in \(\beta(x,y,t)\), travel into/out of the region is ignored, 
and we do not address other realistic processes. While these simplifications permit tractable PDE computations, 
it means our conclusions about how infection waves propagate may not translate directly to real world situations