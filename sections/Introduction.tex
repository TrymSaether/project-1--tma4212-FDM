\section{Introduction}
Mathematical models are essential for the modern world to function. They are tools that modern society could not 
exist without. Specifically numerical models allow us to simulate a multitude of different scenerios, giving us 
insights into the mechanics of our world. One example of this is disease modeling. Accurate modeling of infectious disease spread requires capturing both the dynamics of transmission and how that
transmission propagates across space. While the classical Susceptible Infected Removed (SIR) model offers 
insight into temporal evolution, it does not account for spatial movement of the population. To address 
this limitation, we extend the SIR framework to a reaction--diffusion system, adding diffusion terms that 
approximate individual mobility within a two-dimensional domain. 

Such a PDE-based approach, however, raises numerical challenges: the diffusion and reaction components 
interact in ways that can destabilize straightforward explicit methods. To overcome this, 
we employ a finite difference method (FDM) in space combined with a modified Crank--Nicolson scheme in 
time that handles the diffusion term implicitly and the reaction term explicitly. 

This report focuses on the theoretical underpinnings of this approach, assessing consistency and stability 
through classical analyses, then validates the method numerically. By exploring varying initial distributions 
of infection, different parameter regimes, and dynamic infection rates, we illustrate how the method captures 
key spatiotemporal patterns in disease spread while preserving computational efficiency and stability.




