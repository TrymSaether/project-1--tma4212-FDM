Mathematical models are essential for understanding infectious disease dynamics and predicting outbreak spread. 
The Susceptible-Infected-Removed (SIR) model describes disease transmission but lacks spatial components, 
which is critical for real-world comparison. To address this, the reaction-diffusion SIR model extends 
the system to partial differential equations (PDEs) by incorporating diffusion terms that model population 
mobility. This introduces computational challenges, requiring an efficient numerical approach for stability 
and accuracy.  

This report applies a finite difference method (FDM) for spatial discretization and a modified Crank-Nicolson 
scheme for time integration, treating diffusion implicitly and reaction explicitly. The objectives are to 
analyze the scheme’s numerical properties, validate it through numerical experiments, and investigate disease 
spread under varying initial conditions and parameter settings.
