\section{Introduction}
Mathematical models are fundamental in the role of understanding the dynamics of infectious diseases, 
helping predict the spread of outbreaks, which in turn is an enormous resource for humanity.

The Susceptible-Infected-Removed (SIR) model is a widely used framework for modeling disease transmission within
a population, modeling interactions between susceptible individuals (S), infected individuals (I), and those who
have recovered or been removed from the system (R). While the SIR model usually describes temporal disease 
dynamics at a single location, it does not account for spatial spread. This is a crucial factor in real world,
where movement and interactions between individuals lead to geographical propagation of infections.
To that end, the reaction-diffusion extension of the SIR model introduces spatial components, where diffusion 
terms describe the mobility of susceptible and infected individuals. This leads to a system of partial 
differential equations making the problem computationally challenging, requiring a fitting numerical method 
to obtain stable and accurate solutions. We employ a finite difference method (FDM) for spatial discretization 
and a modified Crank-Nicolson scheme for time integration, where the diffusion terms are treated implicitly 
while the reaction terms are handled explicitly, balancing stability and accuracy.

Thus the objectives of this report are, firstly, to analyze the numerical properties of the proposed scheme, 
including consistency and stability. Secondly, to implement and validate the method through numerical experiments,
comparing theoretical expectations with computed results. And lastly, to investigate the spread of an infectious 
disease under various initial conditions, diffusion rates, and parameter settings.