




By combining a two-dimensional diffusion operator (for $S$ and $I$) with the usual SIR reaction terms, we capture how local infection spreads spatially. Our chosen domain and parameters illustrate various infection patterns:
\begin{itemize}
  \item \textbf{Small \(\mu_S,\mu_I\)}: keeps population mostly localized, leading to distinct infection clusters.
  \item \textbf{Modified \(\beta(x,y,t)\)}: represents spatiotemporal heterogeneity, e.g.\ higher infection rates in the central region or for certain times.
\end{itemize}
Though we used straightforward explicit Euler in time, the moderate diffusion and relatively small \(\Delta t\) suffice to maintain stability. More advanced time-stepping schemes (like Crank--Nicolson) could also be adapted for faster runs if necessary. Overall, our setup provides a flexible environment to explore how disease waves propagate and to visualize key features of spatio-temporal epidemics.